Here we present \textit{easyReporting}, an R6\footnote{\url{https://adv-r.hadley.nz/r6.html}}\footnote{\url{https://cran.r-project.org/web/packages/R6/index.html}} class\footnote{\url{https://en.wikipedia.org/wiki/Class_(computer_programming)}} helping developers to integrate a reproducible research layer inside their software products, as well as lazy analysts to speed up their report production without learning the \textit{rmarkdown} language.

In such a way, thanks to minimal additional efforts of developers, the end user has available an \textit{rmarkdown} file within all the source code generated during the analysis, divided into \glspl{cc} ready for the compilation.

Once manually edited with comments and descriptions the file can be compiled to produce an enriched document within input data, source code and output results.

A so final document can be attached to the publication of the analysis as supplementary material, helping the interested community to entirely reproduce the computational part of work (figure \ref{fig:rrscheme}). 

\subsubsection{General Description and Initialization}

The class can to be imagined as a schematic representation of the \textit{rmarkdown} file (report), indeed 
its attributes represent the report characteristics, which are typically inserted into the header of the file.
Additionally, our class methods are not only for the attributes manipulations but also for insertion of \glspl{cc}, comments and section titles inside the report.

As any typical class, before of using it, \textit{easyReporting} requires to be initialized with the \lstinline!new! command, passing as mandatory arguments the \textit{path} and the name of the file as \lstinline!filenamepath! and a title as \lstinline!mainTitle!.
Additionally, an \lstinline!author! and the \lstinline!documentType! can be specified.

\begin{figure}[ht]
\centering
\includegraphics[width=\textwidth, keepaspectratio]{img/rr/knitropts.png}
\caption[knitr options]{A schematic table of options available using knitr and rmarkdown packages.}
\label{fig:knitropts}
\end{figure}

When initializing, the class automatically creates the report with the entire specified folder tree, setting up its header and declaring the general options for the rmarkdown file.
If rmarkdown personal options (see figure \ref{fig:knitropts} for a list of available options) are required, before creating an instance of the class, it is possible to use the \lstinline!makeOptionsList! function, and then assigning the output to the \lstinline!optionsList! argument of the class \lstinline!constructor!.

\subsubsection{Class Methods}

The class is provided of several methods for \textit{rmarkdown} \gls{cc} construction.

Once an \textit{easyReporting} instance is available, with \lstinline!mkdTitle! it is possible to insert six levels of titles, by setting the parameters \lstinline!title! and \lstinline!level!.
It is also possible to add natural language comments with \lstinline!mkdGeneralMsg!.

When working with \glspl{cc}, two main choices are available.
The first one gives the possibility to construct a \gls{cc} as additional steps, by using first the \lstinline!mkdCodeChunkSt!, then adding variable assignment and/or function calling with \lstinline!mkdVariableAssignment! or \lstinline!mkdGeneralMsg!, and finally closing the \gls{cc} with\\ \lstinline!mkdCodeChunkEnd!.

In particular, when starting a \gls{cc} with \lstinline!mkdCodeChunkSt!, it is possible to assign a specific \lstinline!optionList! and/or a \lstinline!source.files.list! to be added to that \gls{cc}.

Otherwise it is possible to create an entire \gls{cc} just with \lstinline!mkdCodeChunkComplete! and assigning the entire function call as a \lstinline!message!.
This way of working is really useful with \lstinline!function! creation, where inside a developed function a simple recursive call with parameters assignment can be done as a single \lstinline!message!.




