%As previously mentioned in section \ref{sec:reprres}, in the omics data field, the complexity of the analysis, due to the data high-dimensionality and to the wide range of methodologies to use, has revived an interest in \gls{rr}, because of the difficulties in reproducing third-party scientific findings.

%Basically, the implementation of the \gls{rr} is pretty complex, requiring a combination of multiple tools, each one with a different final scope.

%In this Chapter we illustrate \textit{easyReporting} a novel R package for speeding up the \gls{rr} implementation when analyzing data or when constructing other packages.

It has been claimed that many research findings in omics science are false (or partially false) due to accidental mistakes or mis-usage of methods.
To prevent misleading results it is important to be able to inspect and reproduce the entire data analysis carried out in a unified product.

\gls{rr} consists on making available both the analytic data and the associated code in a manner that other researchers might reproduce the findings. 

In this Chapter we illustrate \textit{easyReporting} a novel R package for speeding up the \gls{rr} implementation when analyzing data or when constructing other packages.

Tool availability at: \url{https://github.com/drighelli/easyreporting}