The complexity of \gls{ngs} data analysis brought the scientific community to develop multiple methodologies and tools for their analysis, reaching in some cases (such as \textit{RNA-seq}, \textit{ChIP-seq}, etc.) standard pipelines for their analysis.
However, despite such efforts, there are specific context that are much less developed and that still require attention.

Nowadays, the challenge moved on the development of statistical and computational methodologies for the information extrapolation from multiple-omics datasets, in order to produce a unified view of the biological characteristics describing the cellular behaviour.

Motivated by these aspects, we have embarked on a path that goes through the analysis of different omics to flow into multiple projects, each one with the common aim of helping the single omics analysis in order to simplify reaching the goal of integrating them in a unified view.

This thesis is focused on the development of three main computational tools (\textit{Ticorser}, \textit{DEScan2} and \textit{IntegrHO}), firstly approaching single omics data analysis and then on their integration.
Additionally, in order to face up the irreproducility problem in the scientific community, a fourth tool (\textit{easyReporting}) is presented.

\gls{tic} is a novel R package aimed to analyze time-course \textit{RNA-seq} data, which still require more attention. It offers multiple methods for differential expression data analysis and provides multiple plots useful to explore and visualize the results at each step of the analysis. Furthermore, it also provides methods for functional integration by annotating genes in pathways and GO-terms.

\gls{descan} is a novel R package for \textit{ATAC-seq} data analysis, one of the emerging techniques for investigating chromatin accessibility, for which very few methods are available. It consists in the following three-step procedure: 1) It identifies candidate regions inside each sample implementing a peak caller; 2) It filters out potential artefacts by aligning the candidate regions between the samples and removing those candidate regions that were not reproducible between samples; 3) It produces a count matrix of regions and samples, useful for differential enrichment between multiple conditions and also for integrating this data type with other omics data, such as \textit{RNA-seq}.

\gls{igro} is a \gls{gui}, written in R and Shiny, aimed to analyze and integrate multi-omics data types. 
It provides a friendly interface to the above-mentioned tools and also incorporates a wide selection of methods and other tools available in the literature. This platform, through an easy point-and-click approach, enables the user to analyse and explore single omic data, such as \textit{RNA-seq}, \textit{ChIP-seq} and \textit{ATAC-seq} and, moreover, it offers the possibility to integrate them at different levels, such as gene-peak annotation and functional annotation methods.
Trying to provide a flexible and easy-to-use tool enjoyable also by non-expert analysts, such as biologists, or beginner analysts.

Finally, since in last decades the scientific community has experienced a deep crisis known as ''irreproducible research'', this thesis presents \textit{EasyReporting}, an R package for an automatic report creation, developed to support the reproducibility of scientific research.  
Thanks to the R6 class paradigm on which is based on, it is easy to use and to extend.

Overall, this work proposes and combines several computational tools for properly analysing, visualizing, comparing, integrating and tracing different omics data types.
The results are illustrated with downloaded data from the literature or from collaboration projects.