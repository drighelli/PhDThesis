The complexity of \gls{ngs} data analysis brought the scientific community to develop multiple methodologies and tools for their analysis, reaching in some cases (such as \textit{RNA-seq}) standard pipelines for their analysis.

Nowadays, the challenge moved on the development of statistical and computational methodologies for the information extrapolation from multiple-omics datasets, in order to produce a unified view of the biological characteristics describing the cellular behaviour.

To account for these aspects we have embarked on a path that goes through the analysis of different omics to flow into multiple projects, each one with the common aim of helping the single omics analysis in order to simplify reaching the goal of integrating them in a unified view.

This thesis is focused on the development of three main computational tools (\textit{ticorser}, \textit{DEScan2} and \textit{IntegrHO}), firstly approaching single omics data analysis and then on their integration.
Additionally, in order to face up the irreproducility problem in the scientific community, a fourth tool (\textit{easyReporting}) is presented.

\textit{Ticorser} (time course RNA-seq data analyser) is a novel R package aimed to analyse time-course RNA-seq data. It offers multiple methods for differential expression data analysis and provides multiple plots useful to explore and visualize the results at each step of the analysis. Furthermore, it also provides methods for functional integration by annotating genes in pathways and GO-terms.

\textit{DEScan2} (Differential Enriched Scan 2) is a novel R package for ATAC-seq data analysis, one of the emerging techniques for investigating chromatin accessibility. It consists in the following three-step procedure: 1) It identifies candidate regions inside each sample implementing a peak caller; 2) It filters out potential artefacts by aligning the candidate regions between the samples and removing those candidate regions that were not reproducible between samples 3) It produces a count matrix of regions and samples, useful for differential enrichment between multiple conditions and also for integrating this data type with other omics data, such as RNA-Seq.

\textit{IntegrHO} (Integration of High-Throughput Omics data) is a Graphical User Interface (GUI), written in R and Shiny, aimed to analyse and integrate multi-omics data types. It provides a friendly interface to the above-mentioned tools and also incorporates a wide selection of methods and other tools available in the literature. This platform, through an easy point-and-click approach, enables the user to analyse and explore single omic data, such as RNA-seq, ChIP-seq and ATAC-seq and, moreover, it offers the possibility to integrate them at different levels, such as gene-peak annotation and functional annotation methods.

Finally, to counteract the irreproducibility inside the scientific community we present \textit{EasyReporting}, an R package for an automatic report creation (easyreporting), developed to address the problem of reproducibility of a computational analysis.  
Thanks to the R6 class paradigm on which is based on, it is easy to use and to extend.

Overall, this work proposes and combines several computational tools for properly analysing, visualizing, comparing, integrating and tracing different omics data types, downloaded from the literature or from collaboration projects.
