To study multiple aspects of cellular behaviour related to different experimental conditions, due to drug treatments, pathologies, diseases, etc, several sequencing technologies have been developed during the last decades.

Starting with Sanger sequencing first and passing through Microarrays technologies then, nowadays with Next Generation Sequencing (NGS) we are able to understand many of biological mechanisms, like protein-chromatin interactions (e.g. ChIP-seq \cite{Park2009}), DNA methylation (Methyl-seq or BS-seq \cite{Frommer1992}), chromatin accessibility (e.g. ATAC-seq \cite{Buenrostro2013}), global transcriptional and translational activities (e.g. RNA-seq \ref{Wang2009}) and 3-D organization of chromatin (e.g. Hi-C \cite{VanBerkum2010}), giving the possibility to study same individual or experimental condition from many different points of view (transcriptomics, epigenomics, etc.) with a very high resolution. Each type of these ''omics'' data explains a different aspect of cellular behaviour. 
To give a comprehensive view of the cell regulatory mechanisms it is necessary not only to perform a single level analysis but also to provide novel statistical and computational models for integrating different omics types within a unified study.

The common steps valid for almost every \gls{ngs} data type starts from a library preparation (depending on the sequencing technique), an amplification of the genomic material and finally the sequencing of the samples.

The typical analysis requires several steps to achieve good quality results, regardless of the specific final aim of the investigation.
A standard analysis starts with a quality control assessment of the raw sequenced fragments (reads), using the software as \textit{FastQC} \cite{Andrews2010}.
After quality control assessment, it is essential for the analysis to map the raw reads on a reference genome. 
During the last years, in order to obtain higher accuracy in read mapping (alignment), several software, such as \textit{STAR} \cite{Dobin2013} and \textit{HISAT2} \cite{Kim2015}, superseded broadly used tools as \textit{TopHat2} \cite{Kim2013}.
Usually, an aligner produces \gls{bam} files\footnote{\url{http://samtools.github.io/hts-specs/SAMv1.pdf}} \cite{Li2009} where all the reads (mapped and un-mapped) are collected with additional fields relevant for further steps of the analysis.
Depending on the final aim of the analysis, \gls{bam} files can be further processed to delete or to split the reads. 
Usually the software used for this scope is \textit{samtools} \cite{Li2009, Li2011}, available also in R programming language with \textit{rsamtools} \cite{Morgan}.
From this point on, the analysis steps are dependent on the sequencing technique and from the experimental investigation.

Here we present the basics about the main sequencing techniques addressed in this thesis work, needed to better understand the outline of next chapters.
Please refer to the following chapters for details about each data type analysis.