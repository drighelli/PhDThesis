The counting step is designed to take a \textit{GenomicRanges} data structure as input, where additional informations for the score and the number of samples for each peak are stored.
Moreover, it requires also the path of the BAM/BED files within the reads to count for each peak in the \textit{GenomicRanges}.

This step counts the number of reads present for each region in each sample.
Indeed, it produces as result a matrix of counts where on the reads there are the regions and on the columns the samples.

In order to keep trace of all the information associated to the regions, it produces a \textit{SummarizedExperiment} \cite{SummExp} data structure, which gives the possibility to access the \textit{GenomicRanges} data structure associated to the peaks with the \textit{rowRanges} method and to access to the count matrix with the \textit{assays} method.

The choice to produce a count matrix is dictated by the versatility of this data structure, useful not only for the differential enrichment of the regions between multiple conditions, but also for integrationg this -omic type with other -omics.