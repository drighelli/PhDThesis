
Epigenetic, as shown in introduction (cite), is a pretty wide and complex field, and the sequencing technology to adopt depends on the biological question under investigation.

Some studies \cite{Koberstein2018, Auerbach2009} demonstrated the importance of genomewide chromatin accessibility of a broad spectrum of chromatin phenomena activation using sequencing techniques as \textit{Atac-Seq}, \textit{Sono-Seq}, etc.
Even if there are some methods for the analysis of these omic data types, there still is a lack of them, in particular for an emerging omic as \textit{Atac-Seq}.

To address this lack, we decided to create a useful instrument for analysing chromatine regions accessibility data (such as \textit{Atac-Seq}, \textit{Sono-Seq}).
Very often the biological questions to be answered, as for the RNA-Seq, need the comparison of two or more different biological conditions.
Starting from a set of already published \cite{Koberstein2018} scripts, we designed \gls{descan}, a software for helping the analysis of chromatin accession sequencing data.