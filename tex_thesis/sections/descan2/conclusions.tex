In the lack of methodologies for open chromatin region detection and analysis, we developed a novel approach which, compared with very well known tools as \textit{MACS2}, seems to be competitive in the detection of the signal.

\gls{descan} demonstrated to be able to catch not only wide signal but also narrow regions across the samples. 
And with our filtering/aligning step we demonstrated to be able to keep relevant signal producing data structures as \textit{SummarizedExperiment} which are candidates to become standards in the biological data analysis.
With our 3-steps analysis, we put our tool at the top of a pipeline for open chromatin regions data analysis, proposing also a possible candidate for a standard analysis of this data type.

In the next future, we plan to check if other distributions, as \textit{Negative Binomial}, fit better this kind of data and to improve our filtering/aligning step with an additional probabilistic methodology.

