In the lack of methodologies for open chromatin region detection and their analysis, we developed a novel approach which, compared with very well known tools as \textit{MACS2}, seems to be competitive in the detection of the signal.

\gls{descan} demonstrated to be able to catch not only wide signals but also narrow regions across the samples. 
And with our filtering/aligning step we demonstrated to be able to keep relevant signal producing data structures as \textit{SummarizedExperiment} which are candidates to become standards in the biological data analysis.
With our 3-steps analysis, we put our tool at the top of a pipeline for open chromatin regions data analysis, proposing also a possible candidate for a standard analysis of this data type.

The presented case study showed the capability of \gls{descan} to be competitive in the detection, definition and filtering of the signal, but still more comparisons and improvements are needed.

Indeed, in order to validate our obtained results, we plan to compare them with other similar already published approaches, such as \textit{csaw} \cite{Lun2015}, which peak caller, starting from a set of \gls{bam} files, constructs a count matrix of the detected peaks in the samples.
Moreover, to better test our peak caller, we plan to check if other distributions, as \textit{Negative Binomial}, fit better this kind of data and, additionally, in order to improve our filtering/aligning step we want to apply additional probabilistic methodologies, taking into account the experimental design.

Despite these improvements, \gls{descan}, from this preliminary results, already seems to be a powerful and intuitive instrument for \textit{ATAC-seq} signal detection at the top of a more sophisticated analysis pipeline.

