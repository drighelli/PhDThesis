

In the lack of methodologies for open chromatin region detection and analysis, we developed a novel approach which, compared with very well known tools as \textit{MACS2}, seems to be competitive in the detection of the signal, and, because it's newly born, aims to be more poweful and attractive in this field.

We demonstrated to be able to catch not only spread signal, but also small regions across the samples. 
And with our filtering/aligning step we  demonstrated to be able to keep relevant signl producing data structures as \textit{SummarizedExperiment} which are candidates to become standards in the biological data analysis.
With our 3-steps analysis we puts our tool at the top of a pipeline for open chromatin regions data analysis, proposing also a possible candidate for a standard analysis of this data type.

In the next future we plan to check if other distributions, as \textit{Negative Binomial}, fits better this data kind and to improve our filtering/aligning step with additional probabilistic methodology.

