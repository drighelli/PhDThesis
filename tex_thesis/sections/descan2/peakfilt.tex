
In order to filter out false positives peaks, we designed a filtering method (\lstinline{finalRegions}) based on two different thresholds.
A first threshold on the peaks score and a second threshold on the number of samples.

The filtering step is designed to take as input a list of peaks as GenomicRangesList, where each element represents a chromosome.
This is the data structure produced by the peak caller, but, we also developed a method to load peaks produced also by other software like MACS \cite{Zhang2008}, as described in section \ref{sec:descan2addfeat}.

Firstly, using the threshold on the peak's score (\lstinline{zThreshold} parameter), the method filters out the peaks with a score lower than the user-defined threshold value.

Then it extends a 200bp window in both directions of the detected region, computing the overlaps between the samples using the \lstinline{findOverlapsOfPeaks} method (with \lstinline{connectedPeaks} parameter set as \lstinline{merge}), defined by the ChIPpeakAnno \cite{Zhu2010} Bioconductor package.

Based on this idea, the filtering step is developed to filter out those peaks not present in at least a user-defined (\lstinline{minCarriers} parameter) number of samples. In the light of this, the user can decide the minimum number of samples where each peak has to be detected.
We suggest to set the threshold as a mutiple of the number of replicates of the conditions.

