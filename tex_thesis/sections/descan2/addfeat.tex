The package offers some additional features for loading data (i.e. peaks) resulting from other sources, and for manipulating \textit{GenomicRanges} data structure.

To give the possibility to use our pipeline with external peak callers, the function \lstinline!readFilesAsGRangesList! takes as input a directory containing BAM or BED data, to load in \textit{GenomicRangesList} format.
This data structure is useful to store genomic information, as peaks or mapped reads, produced by other software like \textit{MACS2} or \textit{STAR} and, in case of peaks, it is necessary during the \gls{descan} filtering/aligning step.
Additionally to \lstinline!fileType! (BAM, BED, BED.zip) parameter specification, it requires the genome code to use during the file processing.
Moreover, when the input files represent peaks the \lstinline!arePeaks! flag needs to be set to \lstinline!TRUE!.

Furthermore, \gls{descan} provides several functionalities for GenomicRanges data structure
handling. One example is \lstinline!fromSamplesToChrsGRangesList!, which gives the possibility to split a GenomicRangesList by chromosome. 
This procedure could be useful for parallelizing computations on the chromosomes, when common operations on them, between multiple samples, are needed. Assigning a single chromosome to a single computing unit.
Taken as input a \textit{GenomicRangesList} organized by samples, this method returns a list of chromosomes, where each element has a \textit{GenomicRangesList} of samples, containing only the regions associated to the single chromosome.

%[Create figure to better explain the transformation]

Other useful utilities are \lstinline!keepRelevantChrs!, that takes a \textit{GenomicRangesList} and a list of chromosomes and return only the interested chromosomes with a cleaned \textit{genomeInfo} assigned;
the \lstinline!saveGRangesAsTsv! function that saves a tab separated value file starting from a GenomicRanges; the 
\lstinline!saveGRangesAsBed! that save a standard BED file format starting from a GenomicRanges data structure; and the \lstinline!setGRangesGenomeInfo! which, starting from a genome code, sets a specific \textit{genomeInfo} to a \textit{GenomicRanges} object.