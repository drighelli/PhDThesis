The Peak Caller (\lstinline!findPeaks! function) takes as input a set of alignment files (BAM \cite{Li2009} or BED format) with the code of the reference genome (i.e. \textit{mm10} for Mus Musculus version 10) and several additional parameters, useful for the peak detection setup.

The alignement data are stored as \textit{GenomicRangesList} \cite{Lawrence2013}, where each element represents a file. 
In order to facilitate the parallelization of the computations over the chromosomes, the list is re-arranged as a chromosome list of \textit{GenomicRangesList}, where each element represents the file containing just the \textit{GenomicRanges} of the specific chromosome (see section \ref{sec:descan2addfeat}).

For each element of this data structure the algorithm firstly divides each chromosome as bins of \lstinline!binSize! parameter length (default value is 50bp) and then computes the reads coverage on the bins with moving scan windows, spanning from \lstinline!minWin! to \lstinline!maxWin! parameters of \lstinline!binSize! interval.

In order to be able to catch small and spread peaks the algorithm computes the coverage also using windows of two different lengths, that can be defined with \lstinline{minCompWinWidth} and \lstinline!maxCompWinWidth! (defaults values are 5000bp and 10000bp) parameters, computing a matrix of \textit{n} bins and \textit{p} windows.
%\textit{\textbf{for real this part is to construct a sort of background of the signal, to compare the smaller windows to - ask to davide?}}

The coverages matrix is useful to merge contigous regions and to compute a score for each of them, applying a \gls{mle}, assuming a Poisson distribution of the coverages across the windows.

Formalizing: assuming that each window is distributed as a Poisson random variable, we assume to observe the \textit{n} coverages as an IID sequence $X_n$. Thus, the probability mass function is described as:

\[ p(x_i) = \frac{\lambda^{x_i}}{{x_i!}} exp(-\lambda)\]

Where the coverage nature of the data support the Poisson distribution as the set of non-negative integer number and where $\lambda$ is the Poisson parameter to derive with a \gls{mle}, described as the estimator:

\[\lambda_n=\frac{1}{n}\sum_{i=1}^{n}x_i\]

Which corresponds to the sample mean of the \textit{n} observations in the sample.



[describe output as tsv]


