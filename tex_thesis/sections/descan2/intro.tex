The \gls{descan} is an R \cite{Ihaka1996} tool developed for detecting open chromatin regions signal in order to facilitate the differential enrichment of genomic regions between two or more biological conditions.

The package has been implemented using Bioconductor \cite{Gentleman2004} data structures and methods, and it is available on Bioconductor since version 3.7.

\begin{figure}[H]
\centering
\includegraphics[width=\textwidth,height=\textheight,keepaspectratio]{img/descan2/flow.png}
\caption[DEScan2 workflow]{A differential enrichement flow representation. \gls{descan} steps are highlighed in yellow.}
\label{fig:descan2flow}
\end{figure}

The tool is organized in three main steps. 
A peak caller, which is a standard moving scan window that compares the counts within a sliding window, to the counts in a larger region outside the window. It uses a Maximum Likelihood Estimator on a Poisson Distribution, providing a final score for each detected peak.


The filtering step is aimed to determine if a peak is a "true peak" on the basis of its replicability in other samples. This step is based on a double user-defined threshold, one on the peak's scores and one on the number of samples.


Finally, the third step produces a counts matrix where each column represents a sample and each row a peak. The value of each cell is the number of reads for the peak in the sample.

The so produced counts matrix, as illustrated in the figure \ref{fig:descan2flow}, is useful both for doing differential enrichment between multiple conditions and for integrating the epigenomic data with other -omic data types.




