\gls{tic} offers multiple additional features to help the user during the differential enrichment analysis of time-course RNA-Seq data.

\subsubsection*{Gene quantification}

One feature is aimed to give the possibility to quantify the gene expression starting from samples mapping files. 
To provide this functionality we builded up the \lstinline!countBamFilesFeatureCounts! function with the aim to guide and facilitate this operation to the user.
It is based on the \lstinline!featureCounts! method of the \textit{Rsubread} R/Bioconductor package \cite{Liao2013} hard coding some parameters as \lstinline!useMetafeatures! to \lstinline!TRUE! and \lstinline!allowMultiOverlap! to \lstinline!FALSE!.
The user can give as input a list of BAM files and a \gls{gtf} file, choosing  the \lstinline!gtf.attr.type! and \lstinline!gtf.feat.type! to quantify the samples expression on its needs.

\subsubsection*{Results Comparison}

Usually during a differential expression analysis it is pretty common to have the need to compare multiple \glspl{deg} results list.
In order to facilitate this need we developed three different functions for venn diagrams plots, \lstinline!Venn2de!, \lstinline!Venn3de! and \lstinline!Venn4de!.
During this process it is often required not just to show the graphical plot, but it's most important to show the gene lists resulting from the intersections and disjunctions of the venns.
To afford this aim these functions take as input the gene lists to compare and automatically produces as output lists of files within the resulting lists of all the areas of the produced venns.

\subsubsection*{Gene Identifiers Conversion}

The \gls{tic} package exports multiple functions for the gene names manipulation. 
Indeed, it is very common the need to convert \glspl{deg} list from a specific identifier to another.
We developed \lstinline!convertGenesViaMouseDb! and \lstinline!convertGenesViaBiomart! which convert a \glspl{deg} list using the \textit{org.Mm.eg.db} \cite{Carlson2018} for \textit{Mouse} and using the \textit{biomaRt} R/Bioconductor packages for \textit{Human, Mouse} and \textit{Rat}.
Additionally, it's possible to easily attach the resulting list to a \textit{dataframe}, by using the \lstinline!attachGeneColumnToDf! function, which takes care of adding a new column within the mapped identifiers in the right places of the original \textit{dataframe}.

\subsubsection*{Input/Output File}

To speed up the reading and writing of input/output files, \gls{tic} offers two main functions, \lstinline!readDataFrameFromTSV! and \lstinline!writeDataFrameAsTSV!.

In the first case there is only one mandatory parameter, the \lstinline!file.name.path!, even if gives the possibility to change the classical parameters as \lstinline!row.names.col, headed.flag, sep, quote.char!.

While in the second case the function requires the \lstinline!data.frame.to.save! and the \lstinline!file.name.path! where to store the object. 
Also in this case it is possible to set the classical parameters as \lstinline!col.names! and \lstinline!row.names!.

Moreover, we implemented a method for creating a folder path in a recursive way.
COMPLETARE
Of course if a user need to create a single path by itself can simply create it using the \lstinline!dir.create! R function.