Low expressed genes in \textit{RNA-seq} data, always affect the detection of \glspl{deg} \cite{Sha2015} influencing final results.

In order to address this task, \gls{tic} offers multiple statistical methods for low expressed genes filtering. 
The \lstinline!filterLowCounts! method is partially based on the \lstinline!filtered.data!  function of \textit{NOISeq} R/Bioconductor package \cite{Tarazona2011, Tarazona2015}. 
The method gives the possibility to apply four different filtering methods, the \gls{cpm}, the \textit{proportion} and the \textit{Wilcoxon} tests and an our-defined method named \textit{quantile}.

The \gls{cpm} filters out all those genes with a mean expression between the samples lesser than the \lstinline!cpm! parameter threshold and, at the same time, a \gls{cov} higher that the \lstinline!cv.percentage! parameter in all the samples.

The \textit{proportion} test performs the homonym test on the counts, filtering out all those features with a relative expression equal to $cpm/10^6$.

The \textit{Wilcoxon} test filters out all those genes with a median equal to $0$.

Finally, the \textit{quantile} method enables to filter out all those genes which express the counts mean between all the samples higher than the \lstinline!quantile.threshold! argument (default is $99\%$).