\gls{tic} offers three different ways for analyzing time course RNA-Seq data.

Depending on the biological question under investigation we designed three different ways of interrogate the data in a time-course experiment.

Moreover, \gls{tic} offers three different ways for analyzing different biological conditions in a single time point.

To do so, we took advantage of some of the mostly used and well-performing \cite{Costa-Silva2017} R/Bioconductor packages, \textit{DESeq2} \cite{Love2014};\textit{MASigPro} \cite{Nueda2014}; \textit{edgeR} \cite{Robinson2009}; \textit{NOISeq} \cite{Tarazona2012}.

All the selected methods model the \textit{RNA-seq} data counts for each gene as independents Negative Binomial distributions, which has been demonstrated \cite{Robinson2007} to be better suited for this data type.
At the same time, each of them differs for the statistical test implemented, while approaching to the biological question under investigation.

In the following sections we firstly present the Time-Course methods and then the methods for single time point gene differentiation.

\subsubsection{Time-Course DE Method 1 - \textit{LRT-TC}}
The first method (\textit{LRT-TC}) uses a \gls{lrt} to compare two different models in order to extract all those \glspl{deg} that invert their expression expression between the conditions across all the time points.

Exploiting the \gls{lrt}, as implemented in \textit{DESeq2} R/Biocnductor package, we compare two different formulas.
The first one defines the \textit{full} model where we put together the timepoints, the conditions and an interaction term between these two variables, while the second one is a reduced model where the interaction term is removed:

In so doing we are able to catch all the genes inverting their expression across the conditions along the time-course experiment. 

\[LRT \sim \frac{times+conditions+times:conditions}{times+conditions}\]


\subsubsection{Time-Course DE Method 2 - \textit{LRT-T}}
The underlying idea of the second method is the same of the first one, where the difference, here, is to remove from the \textit{reduced} formula, not only the interaction term, but also the \textit{conditions} variable.
In such a way we are able to extract all those \glspl{deg} that have different expression profiles between the conditions across all the time points.

The first formula here defines the same \textit{full} model of the first method, while the second one is the reduced model where only the times appear:

\[LRT \sim \frac{times+conditions+times:conditions}{times}\]


\subsubsection{Time-Course DE Method 3 - \textit{LRT\_NOInteraction}}
Using always the \textit{DESeq2} \gls{lrt} we defined a third method for the 
identification of \glspl{deg} that have different expression between the conditions across all the time points, but that maintain the same profile in both conditions.

Here the \textit{full} model defines the time points and the conditions variables without taking into account the interaction term, while the second the \textit{reduced} model presents only the time point variable:

\[LRT \sim \frac{times+conditions}{times}\]


%\subsubsection{Time-Course DE Method 4 - \textit{nextMASigPro}}
%
%The fourth method takes advantage of the \gls{glm} with Negative Binomial distribution as defined in the \textit{maSigPro} R/Bioconductor package.
%This method, unlike the previous ones, allows to detect all the \glspl{deg} showing any kind of differences between the conditions across all the time points.
%Indeed, as suggested by the \textit{maSigPro} authors is a good norm to cluster the genes to better understand which is their singular behaviour.


\subsubsection{Single DE Methods}

To account for fixed time point experiment we implemented functionalities for helping also the exploration of this aspect, using three different methodologies.
 
%Our package offers a way to analyze the differences between the conditions at single time point, offering three different methodologies.

By using the \lstinline!differentiateConditions! function, it is possible to choose between the \textit{edgeR}, \textit{DESeq2}, \textit{NOISeq} and \textit{NOISeqBio}.

In case of \textit{edgeR} we decided to use the \textit{Quasi-Likelihood} method for the differential expression.
While when using \textit{DESeq2} for this specific case we choose the \textit{Wald} test, as authors suggest.

The \textit{NOISeq} package offers the possibility to discriminate between \textit{biological} and \textit{technical} replicates, computing a posterior probability in both cases, but applying different hypothesis tests.




