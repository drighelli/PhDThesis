Thanks to our previous experiences \cite{russo2015advantages} in developing \gls{gui}, we noticed the growing interest by the scientific community in using interactive software.
This interest is leaded by multiple motivations, such as the need to analyze data very fast or the lack of time in learning programming languages and terminal-line based tools.

Whatever is the motivation, several tools \cite{Poplawski2016} have been proposed during last years but they are oriented to analyze one specific omics or, when designed for multi-omics, they are pipeline oriented. 

In this chapter we introduce \gls{igro}, our web based platform for multi-omics data analysis and integration in a \gls{rr} spirit.



Even if the bioinformatics community has massively moved on the development of novel statistical and computational methods for multi-omics data integration, part of the scientific community is still anchored to the single-omics analysis side.
Indeed, even if really useful, it is still very common to read published papers taking into account analysis of single-omics data without taking into account possible integrated solution with other omics data types.

Of course it is not so simple to afford for multiple omics data experiments, but, nowadays, the internet swarms of public datasets.
In particular when looking at public biological data banks, such as \gls{geo}\footnote{\url{https://www.ncbi.nlm.nih.gov/geo/}} \cite{Services2007} or \gls{tcga}\footnote{\url{https://cancergenome.nih.gov/}} \cite{TheCancerGenomaAtlas2013a}, where it is possible to retrieve as many data as needed.

On the other side, if there are not so much papers publishing integrated analysis, it is also difficult for the analysts to retrieve the right methodologies for the multi-omics data analysis and their visualization. 

Also on these considerations, we decided to provide the scientific community of a novel easy-to-use instrument which starting from the single-omics data analysis, guides the user to multiple ways of integrating multi-omics data type.



