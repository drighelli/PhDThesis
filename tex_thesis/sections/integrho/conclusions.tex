We presented \gls{igro}, a web-platform for multi-omics data analysis, integration and their visualization, tracing each step performed by the user by storing input/output data produced and saving executed data inside an \gls{rmd} file, which can be edited and compiled through the interface itself.

\gls{igro} is a module-based software allowing easy expandability with other methodologies by implementing additional shiny modules.
Indeed, a platform as \gls{igro} can be expanded with several functionalities, but in particulare we are focusing on the implementation of methodologies for high-dimensionality samples integration methods, such as \textit{mixOmics} \cite{Rohart2017} and \textit{MoFa} \cite{Argelaguet2018}.

Moreover, to allow the user to account for better quality analysis of \textit{ChIP-seq} data, we are working on novel interfaces for the normalization and on methodologies for accounting for the right normalization to use.

Additionally, \gls{igro} at the moment traces the code and stores the input/output data into caching database files, in order to construct an \gls{rmd} file live editable from a dedicated interface.
To improve \gls{igro} \gls{rr} layer we would like to implement additional methodologies, such as the graph construction of executed modules, in such a way the user will have also a graphical representation of the final enriched document, that can be presented as a graphical abstract of the analysis.

In general, at the moment \gls{igro} allows to deeply investigate single-omics data and, at the same time, it allows for an integration at functional analysis and a peak-gene annotation levels, offering various modules for data viasualization.
But, in order to promote a wider overview of integration methodologies,it still needs some work to do.
