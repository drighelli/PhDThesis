We presented \gls{igro}, a web platform for multi-omics data analysis, integration and their visualization.
Moreover, it traces each step performed by the user by storing input/output data produced and saving executed data inside an \gls{rmd} file, which can be edited and compiled through the interface itself.

Thanks to its modularity \gls{igro} allows easy expandability with other methodologies.
Indeed, we plan to implement other functionalities for project manipulation and tracing. 
Indeed, to extend the \gls{rr} layer we planned to implement methodologies for graph construction of executed modules, in order to produce also a graphical representation of the final enriched document.

Moreover, to deeper address the multiple omics integration phase, we  are working on the implementation of methods like mixOmics and MoFa, which gives a graphical response of the samples or the features, when working with high-dimensional datasets.