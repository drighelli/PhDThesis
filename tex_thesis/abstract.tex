\section*{Abstract}
{\setlength{\parindent}{0cm}
Massive parallel sequencing technologies are producing a vast amount of genome-wide data about cells, tissues and model organisms, useful to understand many of biological mechanisms, like protein-chromatin interactions (e.g. ChIP-Seq), DNA methylation (Methyl-Seq or BS-Seq), chromatin accessibility (e.g. Atac-Seq), global transcriptional and translational activities (e.g. RNA-Seq) and 3-D organisation of chromatin (e.g. Hi-C), giving the possibility to study same individual or experimental condition from many different points of view (transcriptomics, epigenomics, etc.) with a very high resolution. Each type of these omics data explains a different aspect of cellular behaviour.
In order to extrapolate relevant information from each omics, it is required to develop specific statistical methodologies for single data analysis and, at the same time, computational methodologies for handling huge amount of data.
But to give a comprehensive view of the cell regulatory mechanisms, it is necessary not only to develop a single omics analysis methodologies but also to provide novel statistical and computational models for integrating different omics types within a unified study.

This thesis is focused on the development of three main computational tools (\textit{ticorser}, \textit{DEScan2} and \textit{IntegrHO}), allowing data analysis and integration of multiple next-generation sequencing experiments.
Additionally, a fourth tool (\textit{easyReporting}) for reproducible computational research is presented.

\textit{Ticorser} (time course RNA-seq data analyser) is a novel R package aimed to analyse time-course RNA-seq data. It offers multiple methods for differential expression data analysis and provides multiple plots useful to explore and visualize the results at each step of the analysis. Furthermore, it also provides methods for functional integration by annotating genes in pathways and GO-terms.

\textit{DEScan2} (Differential Enriched Scan 2) is a novel R package for ATAC-seq data analysis, one of the emerging techniques for investigating chromatin accessibility. It consists in the following three-step procedure: 1) It identifies candidate regions inside each sample implementing a peak caller; 2) It filters out potential artefacts by aligning the candidate regions between the samples and removing those candidate regions that were not reproducible between samples 3) It produces a count matrix of regions and samples, useful for differential enrichment between multiple conditions and also for integrating this data type with other omics data, such as RNA-Seq.

\textit{IntegrHO} (Integration of High-Throughput Omics data) is a Graphical User Interface (GUI), written in R and Shiny, aimed to analyse and integrate multi-omics data types. It provides a friendly interface to the above-mentioned tools and also incorporates a wide selection of methods and other tools available in the literature. This platform, through an easy point-and-click approach, enables the user to analyse and explore single omic data, such as RNA-seq, ChIP-seq and ATAC-seq and, moreover, it offers the possibility to integrate them at different levels, such as gene-peak annotation and functional annotation methods.

Finally, to counteract the reproducibility of scientific research we present \textit{EasyReporting}, an R package for an automatic report creation (easyreporting), developed to address the problem of reproducibility of a computational analysis.  
Thanks to the R6 class paradigm on which is based on, it is easy to use and to extend.

Overall, this work proposes and combines several computational tools for properly analysing, visualizing, comparing, integrating and tracing different omics data types, downloaded from the literature or from collaboration projects.
}